\begin{abstract}
    \ac{hgt} is a mechanism by which organisms (mainly prokaryotes) can share genetic material outside of inheritance.
    \ac{hgt} has proven to have significant effects on bacterial genome evolution, allowing for increased genetic diversity and advanced niche adaptation.
    \ac{crspc} is an adaptive immune system in prokaryotes that has garnered a lot of research attention recently, largely due to it's applications in gene editing.
    Due to the nature of how it  works, using guide RNA to cut DNA, \ac{crspc} systems have been thought to affect rates of \ac{hgt}.
    Effort has mostly been focused on how \ac{crspc} systems affect the mechanisms of \ac{hgt} and thus little is known about its effects on \ac{hgt} rates.
    This work uses is a network-theoretic approach to better characterize the effects of the presence of \ac{crspc} systems on \ac{hgt} rates within bacterial populations.
    This approaches makes use of phylogenetic methods for estimating \ac{hgt} rates, improving on estimates using methods based on genome composition used previously.
    Understanding the effects of \ac{crspc} on \ac{hgt} may help uncover potential targets for curbing the spread antibiotic resistance genes.
\end{abstract}

