\section{Background}
\subsection{What is CRISPR-Cas?}
\ac{crspc} systems are sets of nucleotide motifs (spacers) interspaced with nucleotide repeats (\ac{crsp}s) and \ac{cas} proteins (with sequences usually adjacent to the \ac{crsp} motifs) that have an adaptive immune function in many bacteria and archaea \citep{crispgen}.
Each nucleotide motif is the result of a DNA molecule that was previously taken up by the host, serving as a marker for the \ac{cas} proteins to degrade any DNA matching the motif \citep{crispgen}.
If a bacterium possessing a \ac{crspc} system is infected with a phage and survives, a motif representative of that phage can be integrated as a spacer.
If the bacterium is reinfected with the same phage strain the spacers will guide \ac{cas} proteins to the invading phage DNA and degrade it, hence adaptive immunity.
Although \ac{crspc} appears to have evolved to degrade viral DNA system, the majority of \ac{crsp} spacers have been found to match bacterial \ac{mge}s with no known viral match \citep{nonvspacer}.\par
As of 2017, over $45\%$ of bacterial genomes analyzed ($n=6782$) appear to contain \ac{crsp} motifs \citep{crispdb}.
Moreover, \ac{crsp} motifs show significant diversity between individual bacteriums as they represent a chronological history of spacer acquisition (usually via viral infection or \ac{mge} ``infection'') for that specific bacterium\citep{crispgen}.
There still exist many bacterial strains, and even entire genera with no \textit{known} \ac{crspc} systems, although they may have simply not been discovered yet \citep{ineqcas,casguild}.
Furthermore, the diversification of CRISPR-Cas systems is driven further by \ac{hgt} acting on \ac{crsp} and \ac{cas} components independently, adding another level of complexity to the propagation of CRISPR-Cas systems \citep{crispgen}.

\subsection{Horizontal Gene Transfer}
\subsubsection{Mechanisms}
\ac{hgt} can be defined as the exchange of genetic information across lineages \citep{lgt}, as opposed to vertical gene transfer between parents and offspring \citep{ihgt}.
It is a source of genetic variation, allowing organisms to adapt quickly by copying a gene with a specific function, rather than evolving it themselves \citep{ihgt,adaevo}.
There are 3 main mechanisms of \ac{hgt}\\
\textbf{Transformation}
Free floating DNA is taken up by a bacterium and incorporated into the genome \citep{lgt}. DNA is not always incorporated successfully even if it taken up.\\
\textbf{Conjugation}
The sharing of genetic material through cell-to-cell bridges, the genes for which are usually carried on a plasmid \citep{conjug}.\\
\textbf{Transduction}
Genes or DNA fragments can be transferred through either lytic or lysogenic bacteriophages \citep{transd}.
Bacterial DNA can be accidentally packaged into the lysogenic phage head during cell lysis and integrate into the next infected host.  \citep{transd}.
Lysogenic phages can take up bacterial DNA flanking the viral sequence and bring it with them to the next host\citep{transd}.\par
It should be noted that \textit{successful} \ac{hgt} requires that a gene be maintained, either by genomic integration or plasmid replication.
Frequently, putatively transferred genes are either lost quickly or diverge quickly due to minimal selective pressure maintaining them \citep{fastlane}.
\subsubsection{Rate Influencing Factors}
The rate of \ac{hgt} in bacteria is constantly in flux\citep{trendbs}.
The more exogenous DNA, higher population density or higher phage density means more DNA is available for transfer \citep{lgt}.
Just like mutation rates, \ac{hgt} rates are also thought to evolve in response to environmental factors or selective pressure \citep{mtrate,hgtrate}.
The metaboliccost or the possibility of receiving toxic or incompatible genes can dis incentivize a cell to produce the machinery required for \ac{hgt} \citep{hgtcost}.
But for bacteria in hospitals, the potential benefit of receiving antibiotic resistance genes can outweigh any potential danger or metabolic cost, inducing increased bacterial competence \citep{hospital}
It has been suggested that genes acquired via \ac{hgt} are often quickly lost since they often confer no advantage to a cell's current selective pressures \citep{fastlane}.
Ultimately \ac{hgt} rates are influenced by a variety of factors related balancing the potential fitness costs and benefits.
\subsection{Phylogenomic Networks}
\ac{hgt} is an important factor in understanding evolution in prokaryotes.
In graph theory a tree is defined as a graph where there is only one path between every pair of nodes.
In phylogenetics this implies there is only one path for genetic material to transfer between organisms, that path being vertical inheritance.
The existence of \ac{hgt} demonstrates that the tree model is clearly an incomplete representation of genetic relationships between bacterial \ac{otu}s.
Genetic material can be transferred outside of reproduction, creating multiple paths through which a gene can exist in two different \ac{otu}s  \citep{lgt}.
The frequency of \ac{hgt} among prokaryotes has lead many to re-evaluate the concept of a ``prokaryotic tree of life'', which ignores these horizontal interactions \citep{netoflife}.
This prompted the idea of a prokaryotic network of life (as opposed to a tree), with edges indicating both vertical and horizontal transfers of genetic material \citep{netoflife}.
Edges can now connect closely or distantly related \ac{otu}s if \ac{hgt} has occurred between them.
\subsection{Detection}
While understanding that \ac{hgt} is important to bacterial evolution and networks provide a useful theoretic framework to study it, constructing such networks is not trivial.
Previously researches have used methods based on gene composition (GC content, codon usage, di/tri-nucleotide frequency) to detect transferred genes.
These methods are often inadequate as there are several reasons why gene composition may differ in one region vs the rest of the genome that are not related to transfer.
Further they often cannot discriminate \ac{mge}s which may be fasely annotated as transfers.
Phylogenetic methods, which rely on recognizing discordance between gene trees and species trees, are currently considered the best for inferring \ac{hgt} events.
If a gene tree is found to have a significantly different topology from a species tree, this difference may be the result of an \ac{hgt} event \citep{hgterr}.
Generally phylogenetic methods are preferred for multiple reasons:
\begin{itemize}
    \item Can make use of multiple genomes at once \citep{ihgt}
    \item Require explicit evolutionary models, which come with their own framework for hypothesis testing and model selection \citep{ihgt}.
    \item \ac{hgt} events identified by parametric methods are often found by phylogenetic methods as well \citep{ihgt}.
    \item In recent years, the requirements of computing power and  multiple well sequenced genomes for phylogenetic methods have become easier and easier to meet \citep{ihgt}.
\end{itemize}
While detecting \ac{hgt} events with high degrees of certainty is still difficult, much progress has been made in recent years,especially using phylogenetic methods \citep{ihgt}.
Events that may lead to false diagnosis of \ac{hgt} are: incomplete lineage sorting, gene duplication followed by loss in one of the descendant lineages or homologous recombination \citep{ihgt,hgterr}.

\subsection{How Does CRISPR Affect HGT?}
\subsubsection{Interference Mechanisms}
\ac{crspc} systems have also been found to interfere with transformation-mediated \ac{hgt}, by degrading foreign DNA taken up by a cell \citep{climtrans}.
They have been shown to interfere with conjugation by targeting genes on conjugative plasmid \citep{staphlim}.
They have also been shown to interfere with transduction by creating immunity to phage infection \citep{staphlim}.
Thus it been hypothesized that lower rates of \ac{hgt} will be observed in bacterial strains with \ac{crspc} systems than without \citep{staphlim}.
\subsubsection{Complexities And Costs Of CRISPR-Cas Systems}
Since antibiotic resistance genes are often transferred on plasmids maintaining a \ac{crspc} systems can present a large opportunity cost, especially in environments like hospitals or trees \citep{hospital}.
\ac{crspc} systems incur a metabolic cost, as \ac{cas} proteins, guide RNAs and  spacer acquisition proteins must all be expressed consistently\citep{crispgen}.
Much like the human immune system, \ac{crspc} systems can have off-target effects, sometimes affecting \ac{hgt} \citep{acqorres}.
While resisting lytic phage infection clearly provides some fitness benefit, \ac{crspc} has also been shown to help resist prophages which can provide super-infection immunity or reduce competitor populations \citep{acqorres,transhgt}.
It has also been shown that spacers targeting a bacterium's own DNA can be acquired, leading to an auto-immune response \citep{selfcrisp}.
As \ac{crspc} systems prevent phage infection, mechanisms, denoted as anti-\ac{crsp}s, have evolved in certain phages making them immune to \ac{crsp} \citep{acqorres}.
This has a two-fold effect, as it can increase the susceptibility of the host to infection reducing the fitness benefit of \ac{crsp}, but it can also increase transduction-mediated \ac{hgt} \citep{acqorres}.
\subsubsection{Balancing the Cost of \ac{crsp} and \ac{hgt}}
Due to the myriad of fitness costs associated with consistently expressing \ac{crspc} systems, bacteria have appeared to develop strategies to mitigate these costs.
It has been posited that \ac{crspc} systems need only be present in some proportion of a population as they can be horizontally transferred themselves between members\citep{crisprlgt}.
This allows populations to maintain phage immunity while isolating the metabolic cost to only a few organisms \citep{acqorres}.
It should also be noted that \ac{crspc} genes are not necessarily constitutively transcribed, potentially allowing bacteria to tune their \ac{crspc} to suit their selective pressures \citep{acqorres}.
The presence of \ac{crspc} systems have also been shown to actually enhance \ac{hgt} at a population level via transduction by reducing total phage abundance \citep{transhgt}.\par
A bioinformatic analysis has shown increased levels of gene insertion and deletion as in Firmicutes with \ac{crspc} systems compared to closely related outgroups without \citep{athena}.
The effects of \ac{crspc} systems on \ac{hgt} rates are highly complex, owning in no small part to the broad range of these effects, how \ac{crsp} activity can be modulated and the transfer of \ac{crsp} systems within a population \citep{acqorres}.
Taking a systematic approach may help elucidate the dynamics between \ac{crsp} system presence and \ac{hgt} rate.

