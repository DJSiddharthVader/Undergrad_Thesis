\section{Discussion}
\subsection{Gene Indel Rates are Different for \ac{crsp} and Non-\ac{crsp} \ac{otu}s}
For most genera, the gene indel rate for non-\ac{crsp} genera is larger than for \ac{crsp} genera.
This is in-line with literature surrounding the mechanisms of \ac{hgt} as \ac{crspc} systems are meant to stop the integration of foreign DNA into the bacterial genome.
Despite this, the mean node degree of \ac{crsp} and non-\ac{crsp} \ac{otu}s is relatively similar across genera.
One reason for this discrepancy may be that there are often more non-\ac{crsp} \ac{otu}s than \ac{crsp} \ac{otu}s, thus more genes are exchanged between all non-\ac{crsp} \ac{otu}s, but each non-\ac{crsp} \ac{otu} transfers genes at a similar rate to \ac{crsp} \ac{otu}s.
One possible explanation for certain genera having very high gene indel rates for \ac{crsp} \ac{otu}s may be that it is an efficient way to acquire new spacers.
\ac{crspc} systems may enhance \ac{hgt} to preemptively acquire new spacers from the environment in response to environmental phage density.
\subsection{Phylogenomic Networks Have Low Assortativity}
There seems to be significant \ac{hgt} between \ac{crsp} and non-\ac{crsp} \ac{otu}s, with no clear clustering, assortativity or modularity among most of the networks examined.
\ac{crspc} systems do not appear to have a segregating effect on the network, but do appear to have a population level effect of decreasing the rate  of \ac{hgt}.
As suggested by \citep{transhgt} \ac{crspc} systems may have a population level effect on \ac{hgt} rate, but it appears to be suppressive in this case, as opposed to theirs.
\subsection{\ac{hgt} Dynamics Vary Across Bacterial Genera}
Despite some trends being observable, the one constant is that there is significant variability between genera with regards to \ac{hgt}.
\ac{hgt} rates can be similar or significantly different between \ac{crsp} and non-\ac{crsp} \ac{otu}s, either can be larger, both can be similar and either large or small.
While the means are similar, the shapes of the distributions of network assortativity and modularity are not homogeneous.
%\subsection{Limitations}
%Some factors that may have introduced bias or error into this analysis:
%\begin{itemize}
%        \item \textbf{Ignored Singletons}: Genes that did not cluster into any families were ignored from future steps, but may have still represented horizontally transferred genes
%        \item \textbf{Ignored Some Gene Families}: For time considerations, only 1500 gene trees were generated for each genus
%        %\item \textbf{Significance Testing}: Samples are not necessarily independent in a network, further node statistics can only be tested for genera with $> 20$ \ac{crsp} and non-\ac{crsp} \ac{otu}s.
%        \item \textbf{Taxonomic Mistakes}: Inconsistencies in taxonomic labelling can result in ignored or misplaced \ac{otu}s.
%        \item \textbf{Multifurcation Error}: Some species trees contained multifurcations, which were resolved randomly to generate a bifurcating tree. Estimating this error by examining variance over different resolutions is possible, but was not done here.
%\end{itemize}
%More insight may be gained by examining specific genera more in-depth, or considering new network based metric for understanding the dynamics of \ac{hgt}.
