\section{Conclusion}
%\subsection{Significance}
This work highlights the large degree of variability in \ac{hgt} rate between bacterial genera.
While there seems to be a broad association of \ac{crspc} systems with decreased rates of \ac{hgt} compared to \ac{otu}s without such systems, prominent exceptions exist.
Further \ac{crsp} or non-\ac{crsp} \ac{otu}s do not appear to transfer genes preferentially to either \ac{crsp} or non-\ac{crsp} \ac{otu}s, respectively.
This lack of clear, significant effects for individual \ac{otu}s may suggest population level effects of \ac{crspc} systems on \ac{hgt}.
This is further supported by the negative relationship between indel rates for Non-\ac{crsp} \ac{otu}s and the fraction of \ac{otu}s in a genus with a \ac{crspc} system.
One of the simplest such explanations is that if there are more \ac{crspc} systems in a population, they are decreasing the overall level of phage and free-floating DNA affecting all \ac{otu}s.
Ultimately, this pipeline provides a fairly straightforward way to study trends in \ac{hgt} for a set of bacterial \ac{otu}s.
Clearly the dynamics of \ac{crsp} and \ac{hgt} warrent further investigation and should be studied within individual genera such as Streptomyces, which have been model systems for studying \ac{crspc} previously.
There are multiple ways to expand this analysis to answer other questions related to the transfer of genes.
Ways to expand the work shown here include inferring transfer direction, using continuous estimation of \ac{crspc} activity, incorportaing ecological data or comparing transfer for functional categories of genes.
%As \ac{hgt} inference methods improve and it becomes possible to discern the direction of transfer with confidence, a whole new set of techniques become available for study.
%Some possible ways to extend this analysis are:
%\begin{itemize}
%    \item \textbf{Inferring direction}: Directed networks have a host of available analytic tools undirected networks do not
%    \item \textbf{Gene function analysis}: Considering the transfer dynamics of different functional classes of genes
%    \item \textbf{Studying movement of \ac{crsp} systems}: Studying how frequently \ac{crsp} systems themselves are transferred from arrays, \textit{Cas} genes
%    \item \textbf{Intergenic comparisons}: Combine any set of fasta files from \ac{otu}s for analyzing transfer dynamics
%    \item \textbf{Continuous \ac{crsp} activity}: Labelling nodes by estimated \ac{crsp} activity (array length, transciptomic data, etc.)
%    \item \textbf{Considering bacterial ecology and environments}: Consider geographically close \ac{otu}s or differences between networks due to environmental factors
%\end{itemize}
%The results of this work will hopefully shed light on how \ac{crspc} systems affect the rate of \ac{hgt}.
%This can help identify new potential strategies for combating the spread of antibiotic resistance.
%This study may also shed light on the fitness effects of \ac{crspc} systems and how they manifest at a population level.\par
%If one is interested in studying the transfer patterns of a specific grouping of genes, either by function, common structural motifs, sequence composition, expression pattern this type of analysis is highly suitable.
%For example, the transfer of CRISPR systems themselves is something largely unstudied.
%Networks can be constructed from Cas and Cfp1 genes as well as identified \ac{crsp} arrays to estimate how often \ac{crspc} systems themselves move around communities.
%Networks constructed from ribosomal genes can be used as a reference point for what very little transfer looks like.
