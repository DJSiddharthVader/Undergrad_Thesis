\begin{abstract}
    \ac{hgt} is a mechanism by which organisms (mainly prokaryotes) can share genetic material outside of inheritance.
    \ac{hgt} has proven to have significant effects on bacterial genome evolution, allowing for increased genetic diversity and niche adaptation.
    \ac{crspc} is an adaptive immune system in prokaryotes that has garnered much research attention due to it's application as a gene editing tool.
    While much of the focus on \ac{crspc} systems has been related to this application, \ac{crspc} has been shown to have a highly complex interaction with \ac{hgt}.
    Effort has mostly been focused on how \ac{crspc} systems affect the mechanisms of \ac{hgt} and little is currently known about the effects of \ac{crspc} on \ac{hgt} rate, both for individuals and on a population level.
    Proposed here is a network-theoretic approach to further the understanding of the effects of the presence of \ac{crspc} systems on \ac{hgt} within a population.
    Network theory has already been applied to better model evolution and relatedness among bacteria, accounting for \ac{hgt} which traditional phylogenetic methods ignore.
    This network-theoretic approach allows for study of \ac{crspc} effects on individual bacteria as well as population level effects on \ac{hgt}.
    Understanding the effects of \ac{crspc} on \ac{hgt} may help develop strategies to curb spreading antibiotic resistance, understanding bacterial evolution and extend the functionality of \ac{crspc} gene editing systems.
\end{abstract}

