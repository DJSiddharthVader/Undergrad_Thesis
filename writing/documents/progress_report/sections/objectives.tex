\section*{\huge Objectives}
Using sequenced genomes, the goal of this project is to construct phylogenetic networks for all strains within sets of genera with and without \ac{crspc} systems.
Ultimately the goal of this project is to examine the relationship of \ac{hgt} rates and the presence of \ac{crspc} systems, using a network theoretic approach. The following sets of comparisons will contribute to the understanding of this relationship:
\paragraph*{Within Network Comparisons}%2 sentences
For genera with strains containing \ac{crsp} and Non-\ac{crsp} species, comparing the network dynamics of those sets of nodes across genera will elucidate if \ac{crspc} systems affect the \ac{hgt} rates or the association patterns of individual \ac{otu}s.
\paragraph*{Between Network Comparisons}%2 sentences
Networks created from genera with no known \ac{crsp} system containing strains (nc-networks) will be compared to mixed networks, containing strains both with and without \ac{crsp} Systems.
This will help determine whether the presence of \ac{crsp} nodes can affect \ac{hgt} network dynamics of \ac{otu}s other than themselves.
A simple example may be that if mixed networks show more overall transfers across the network than nc-networks, \ac{crsp} containing \ac{otu}s may be increasing \ac{hgt} among closely related Non-\ac{crsp} \ac{otu}s.
\paragraph*{Gene Indel Rates Vs. Network Statistics}%2 sentences
Comparing insertion and  deletion rates independantly can help further specify what mechanisms may be responsible for trends observed in network statistics.
If a mixed network is found to be density connected, but also shows a deletion bias, this may imply that most of the genes being transferred may not confer a fitness advantage.
