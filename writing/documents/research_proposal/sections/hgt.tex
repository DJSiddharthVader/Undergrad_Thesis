\section{Horizontal Gene Transfer}
%Broad intro
\ac{hgt} can be defined as the exchange of genetic information across lineages\citep{lgt}.
The word horizontal is in contrast to what can be referred to as "vertical" gene, between parents and offspring\citep{ihgt}.
\ac{hgt} is often a source of novel adaptations, allowing organisms to respond to selective pressures much more quickly than having to evolve new functions in genes themselves\citep{ihgt,adaevo}.\par
\subsection{Mechanisms}
\paragraph{Transformation}
Transformation is the uptake of free floating exogenous DNA by a bacterium and incorporates it into it's genome\citep{lgt}.
Many factors can influence the competency (capability of transformation) of bacteria naturally, such as DNA damage, selective pressures, cell density and multiple methods have been found to induce competency for experimental purposes (cloning)\citep{natcomp}.
\paragraph{Conjugation}
Conjugation is the sharing of genetic material through cell-to-cell bridges, usually carried on either a self-transmissible or non-self-transmissible plasmid\citep{conjug}.
\paragraph{Transduction}
Transduction is the transfer of genes between bacteria through a bacteriophage\citep{transd}.
When a donor cell infected by a phages is lysed, the lysed bacterial fragments can accidentally be incorporated into the phage head\citep{transd}.
When the phage infects a new bacterium the lysed donor fragments are released into the recipient cell, where they can recombine in to the genome\citep{transd}.
While random (general) gene fragments can be transferred as motioned above another type of transduction exists for lysogenic phage\citep{transd}.
Lysogenic phage incorporate themselves into specific regions of a bacterial genomes\citep{transd}.
When they excise themselves they can accidentally incorporate bacterial DNA flanking the incorporated phage DNA and bring it with them to the next page target\citep{transd}.\par
It should be noted that \textit{successful} \ac{hgt} requires that a gene be maintained, either by genomic integration or plasmid replication.
Frequently putatively transferred genes are either lost quickly after transfer or evolve with little functional constraint, as no selective pressure is maintaining them (presumably)\citep{fastlane}.
\subsection{Rate Influencing Factors}
The rate of \ac{hgt} in bacteria is constantly in flux, in part due to the amount of DNA available for transfer\citep{trendbs}.
If there are low levels of exogenous DNA, low population density or low phage density, reduced \ac{hgt} will be observed as less DNA available for transfer\citep{lgt}.
But just like mutation rates, \ac{hgt} rates are thought to evolve in response to environmental factors or selective pressure\citep{mtrate,hgtrate}.
For example, for strains of bacteria in hospitals, the potential benefit of receiving antibiotic resistance genes via \ac{hgt} may far outweigh any potential danger or metabolic cost, inducing a response increasing a bacterium's uptake of foreign DNA.\citep{hospital}
There are clear metabolic costs for \ac{hgt}, as host machinery to allow competency and conjugation are not trivial to synthesize\citep{hgtcost}.
Further, conjugation and transformation are not discriminatory processes, so DNA encoding for toxic products, having sub-optimal codon distribution or incompatible GC content may be taken up, but cannot be successfully incorporated or consistently expressed\citep{hgtcost}.
Conjugated plasmids may also be incompatible with a host due to the replication machinery required by the plasmid\citep{plasincom}.
In fact it has been suggested that genes recently acquired via \ac{hgt} are often quickly lost, having been lost for conferring no advantage or for conferring a specific advantage, that was lost with the removal of whatever selective pressure was maintaining it\citep{fastlane}.
Ultimately \ac{hgt} rates are influenced by a variety of factors, related to barriers and fitness costs/benefits associated with the genes being gained.
\subsection{Pan-genomes}%pang,toolpan
As sequencing costs has decreased, re-sequencing of strains and sequencing of many similar strains has grown drastically.
From this different strains of the same species were compared yielding interesting results: many genes are not found in most of the strains sequenced\citep{toolpan}.
This has lead to the concept of a pan-genome, the sum total of all unique genes among a set of strains\citep{pang}.
A pan-genome has two parts: a core genome consisting of genes common to all strains in a species and an accessory genome, consisting of unique genes present in any of the strains\citep{pang}.
In \ac{ecoli}, it appears that as the total number of strains sequenced increases, the total number of unique genes increases logarithmically, meaning more unique genes are being identified with every new strain sequenced\citep{ecopan}.
These accessory genes are prime candidates for \ac{hgt}, as there are strains known not to have them and may provide some niche-specific adaptation, such as antibiotic resistance\citep{pang}.
The accessory genome can be considered a genetic toolbox that strains have access to through \ac{hgt}, although this access is also limited by barriers to \ac{hgt} (ex: distance, genetic incompatibility etc.).
Pan-genomes can further be categorized as open, if they appear to be expanding, adding more genes from more distant \ac{otu}s or closed, with the total number of unique genes plateauing as more strains are sequenced\citep{pang}.
\subsection{Applications}
While the vast majority of \ac{hgt} has been observed to occur between prokaryotes, cases have been identified between prokaryotes and eukaryotes.
One particular case allowing a beetle to colonize coffee beans after (presumably) gaining a gene from a bacterial strain colonizing it's midgut\citep{beetle}.
This has become an issue as this beetle is considered a pest, estimated to be the cause of over $20$ million USD loses to rural farming families internationally\citep{beetle}.\par
Another much more important reason for studying \ac{hgt} is that it is antibiotic resistance genes have been shown to transfer frequently\citep{amrhgt}.
The transfer of antibiotic resistance genes is so prevalent that the term resistome has be coined to refer to the set of resistance genes that an organism can acquire via \ac{hgt}\citep{amrhgt}.
Understanding the dynamics of \ac{hgt} and ways to limit or inhibit it specifically may prove integral to resolving the issue of the decreasing range of antibiotic effectiveness\citep{amrhgt}.

