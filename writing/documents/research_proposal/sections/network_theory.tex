\section{Network Theory}
Network theory is an extension of graph theory, a branch of mathematics concerning the properties of "graphs".
Graph in this context refers to a set of nodes and a set of edges between those nodes, with edges typically representing some kind of relationship between those nodes\citep{netgen}.
Network theory focuses on modeling interactions using graphs and applying tools built to analyze graphs to gain an understanding of networks and how they function.\par
Consider a social networking site like Facebook, are people more likely to be friends with people who have a similar number of friends?
Modeling this with a network, nodes represent users and an edge between nodes represent whether two users are Facebook friends.
To answer the above question the assortativity of the network can be calculated.
Assortativity is a measure of the network's nodes preference to form edges with more similar nodes\citep{netgen}.
Similarity here refers the difference in the number of edges connected to each node, \textit{i.e.} the number of friends either user has.
If a network has a large assortativity value, it means that similar nodes do connect to each other more often than different ones.\citep{netgen}.
Thus our question can be reformed through the lens of network theory as "Does a network constructed from Facebook user data have high assortativity?"\\\\
While the above example is a simple network, this model can be further extended through:
\begin{itemize}
    \item Adding directions to edges (from one node to another)
    \item Adding weights to edges (often representing data about node interactions)
    \item Adding attributes to nodes themselves (binary,discrete or continuous)
\end{itemize}
An example of a directed, weighted biological network with low assortativity be a gene expression network (nodes:genes, edges:transcription level correlations) with few transcription factors which each modulates expression of multiple unrelated genes.
For this project, nodes will represent \ac{otu}s and edges will represent genetic exchange between those \ac{otu}s, whereas in a normal phylogenies, edges only represent taxonomic relationships.
Despite the complexity of \ac{hgt}, network theory allows a flexible theoretic framework to analyze these interactions that are normally ignored by traditional phylogenetic methods.

