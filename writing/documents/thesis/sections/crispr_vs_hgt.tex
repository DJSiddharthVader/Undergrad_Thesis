\section{Do CRISPR Systems Affect Horizontal Gene Transfer?}
Yes.
\subsection{Interference Mechanisms}
Since \ac{crsp}s have been shown to be capable of interfering with conjugation (conjugative plasmid specific spacers) and transduction (phage immunity), it has been hypothesized that lower rates of \ac{hgt} will be observed in strains with \ac{crspc} systems \citep{staphlim}.
\ac{crspc} systems have also been found to interfere with transformation-mediated \ac{hgt}, by degrading foreign DNA taken up by a cell \citep{climtrans}.
\subsection{Complexities And Costs Of CRISPR-Cas Systems}
As noted above, \ac{crspc} systems have been shown to interfere with plasmid conjugation in \ac{sau} by integrating a spacer targeting the \textit{nickase} sequence, necessary for conjugation in \ac{sau} \citep{staphlim}.
Since antibiotic resistance genes are often transferred on plasmids, this can incur a significant cost, especially in environments with large amounts of antibiotics (ex: hospitals, trees etc.) \citep{hospital}.
\ac{crspc} systems incur a metabolic cost, as \ac{cas} proteins, guide RNAs, spacer acquisition proteins must all be expressed to maintain immunity \citep{crispgen}.
Despite primarily being an immune system, the way \ac{crspc} functions (degrading foreign DNA matching spacers motifs, resisting phage infection) can have off-target effects on \ac{hgt} \citep{acqorres}.
While resisting lytic phage infection clearly provides some fitness benefit, \ac{crspc} has also been shown to resist prophage incorporation \citep{acqorres}.
Prophages can serve as vectors for \ac{hgt}, but they can also provide super-infection immunity, and even reduce competitor bacterial populations through infection \citep{acqorres,transhgt}.
It has also been shown that spacer sequences representative of a bacterium's own chromosomal DNA can be incorporated in to \ac{crsp} array, leading to an auto-immune response where \ac{cas} proteins target native host DNA \citep{selfcrisp}.
As \ac{crspc} systems persist, anti-\ac{crsp} mechanisms have evolved in certain phages, making them immune to CRISPR-Cas, denoted anti-\ac{crsp}s \citep{acqorres}.
This has a two-fold effect, as it can increase the susceptibility of the host to infection, reducing the fitness benefit of CRISPR-Cas, but it can also allow for more transduction-mediated \ac{hgt} \citep{acqorres}.
\subsection{Potential Strategies For Reducing CRISPR-\ac{hgt} Trade-off Costs}
Due to the  myriad of fitness costs associated with consistently expressing \ac{crspc} systems, bacteria have appeared to develop strategies to mitigate these costs.
While \ac{crspc} systems can confer a fitness advantage by providing immunity to phage infection, the fitness cost associated is complex, especially as \ac{crspc} systems  themselves can be transferred  horizontally, either on a plasmid or even through transduction \citep{crisprlgt}.
It has been posited that \ac{crspc} systems need only be present in a few members of a population at once and transferred between members to maintain phage immunity while reducing the cost of constantly maintaining \ac{crspc} systems \citep{acqorres}.
It has been found that the presence of a \ac{crsp} system does not necessarily imply activity of the system, creating new mechanism(s) by which the fitness cost of \ac{crspc} systems can be reduced \citep{acqorres}.
The presence of \ac{crspc} systems have also been shown to actually enhance \ac{hgt} via transduction at the population level by reducing total phage abundance \citep{transhgt}.
The presence of \ac{crspc} systems in Firmicutes have been shown to be associated with increased levels of gene insertion and deletion compared to closely related outgroups, further demonstrating the complexity of this relationship \citep{athena}.
The effects of \ac{crspc} systems on rates of \ac{hgt} are highly complex, owning in no small part to the broad range of \ac{crsp} effects, how \ac{crsp} activity can be modulated and the transfer of \ac{crsp} systems themselves within a population \citep{acqorres}.
Taking a systematic approach may help elucidate the dynamics between \ac{crsp} system presence and \ac{hgt} rate.
