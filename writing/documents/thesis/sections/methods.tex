%Tikz Styles
\tikzstyle{block} = [circle,
                     fill=green!20,
                     text centered,
                     font=\ssmall,
                     text width=1.5cm,
                     rounded corners,
                     draw]
\tikzstyle{slock} = [circle,
                     fill=red!20,
                     font=\ssmall,
                     text width=1.5cm,
                     text centered,
                     rounded corners,
                     draw]
\tikzstyle{flock} = [circle,
                     fill=blue!20,
                     font=\ssmall,
                     text width=1.5cm,
                     text centered,
                     rounded corners,
                     draw]
\tikzstyle{line} = [draw, -latex']
\newcommand{\Checkmark}{$\mathbin{\tikz [x=2.4ex, y=2.4ex, line width=.2ex, green] \draw (0,1) -- (1,0) -- (3,3);}$}
\section{Methods}
\subsection{Summary}
The goal of the project is to create a phylogenetic network from a set of protein fasta files and the corresponding nucleotide sequences.
In this case all full genomes for a given bacterial genus, for analysis of \ac{hgt}.
The workflow is as follows for a single genus:
\begin{enumerate}
    \item Download fasta files
    \item Filter mobile genetic elements from genomes
    \item Cluster all genes into families using Diamond (\% identity $> 80\%$)
    \item Construct a presence/absence matrix of gene families for organisms
    \item Estimate gene family indel rates separately for the CRISPR and non-CRISPR containing genomes using the R package markophylo
    \item Construct a species tree using all 16S rRNA genes that have 1 copy for each member of the genus
    \begin{enumerate}
        \item Align each 16S gene with mafft using defaul settings
        \item Concatenate all alignments together as a nexus file
        \item Build the tree using Mr Bayes (10000 generations, 25\% burn in)
    \end{enumerate}
    \item Construct the gene trees ($\leq 1500$)
    \begin{enumerate}
        \item Only consider families with a gene belonging in at least 40\% of the genomes analyzed (ex: a family with 6 genes in 6 of 15 genomes)
        \item Align each family using mafft with default settings
        \item Build a tree for each alignment using Mr Bayes (10000 generations, 25\% burn in)
    \end{enumerate}
    \item Create 1000 subsets of 50 gene trees through bootstrap sampling
    \item For each subset, use the program HiDe to infer a phylogenetic network from the species tree and the 50 gene trees.
    \item Annotate each network with CRISPR data scraped from the CRISPR-one database.
    \item Using the gene indel rates estimated and the annotated networks examine if there are any trends or effects on the dynamics of \ac{hgt} between organisms with and without CRISPR-Cas systems.
\end{enumerate}
%Summary Figure
\FloatBarrier
\begin{figure}[htb!]
    \centering
    \begin{tikzpicture}
        % boxes
        \node [slock]                       (node1)  {Get Fastas};
        \node [block,below=0.5cm of node1]  (node2)  {Filter MGEs};
        \node [block,below=0.5cm of node2]  (node3)  {Cluster Families};
        \node [block,right=0.8cm of node3]  (node4)  {P/A Matrix};
        \node [block,above=0.5cm of node4]  (node5)  {Get 16S Genes};
        \node [block,above=0.5cm of node5]  (node6)  {Gene Trees};
        \node [block,right=0.5cm of node5]  (node7)  {Species Tree};
        \node [block,below=0.5cm of node7]  (node8)  {Gene InDel Rates};
        \node [block,above=0.5cm of node7]  (node9)  {Infer Network};
        \node [block,right=0.5cm of node9]  (node10) {Annotate CRISPR};
        \node [block,below=0.5cm of node10] (node11) {Compute Stats};
        \node [flock,below=0.5cm of node11] (node12) {Look For Trends};
        %lines
        \path [line] (node1) -- (node2);
        \path [line] (node2) -- (node3);
        \path [line] (node3) -- (node5);
        \path [line] (node5) -- (node7);
        \path [line] (node3) -- (node4);
        \path [line] (node4) -- (node8);
        \path [line] (node7) -- (node8);
        \path [line] (node3) -- (node6);
        \path [line] (node6) -- (node9);
        \path [line] (node7) -- (node9);
        \path [line] (node9) -- (node10);
        \path [line] (node10) -- (node11);
        \path [line] (node11) -- (node12);
        \path [line] (node8) -- (node12);
    \end{tikzpicture}
    \caption{Diagram of the worfkflow for a single genus}
\end{figure}
\FloatBarrier
\subsection{Data Collection}
Protein and nucleotide fastas of all CDS sequences were downloaded from NCBI RefSeq.
\ac{crsp} annotations of \ac{cas} and Cfp proteins from the \ac{crsp}one tool from Zhang and Ye will be used to assess the presence of \ac{crsp} systems \citep{ineqcas}.
%and the \ac{crsp}db (along with a python script) is used to annotate genera as being mixed (containing strains with and without \ac{crspc} systems) or Non-\ac{crsp} (containing no strains with a \ac{crspc} system) \citep{crispdb}.
\subsection{Gene Presence/Absence Matrix}
In order to use the program markophylo to estimate indel rates, a \ac{pa} matrix of gene families and organisms and a species tree are required.
First any genes classified as \ac{mge}s (from NCBI annotations) are removed.
Next genes are grouped into families by reciprocal BLAST hits and single link clustering.
The remaining unclassified genes are compared to the NCBI non-redundant database with BLAST to check if they are genes, and if they are then they are considered their own family with one member.
The \ac{pa} matrix is constructed as follows, for each \ac{otu} a binary vector is created, where each entry represents a gene family and a 1 indicates that that \ac{otu} contains at least 1 gene in that family.
This is repeated for all \ac{otu}s, creating a $G \times O$ binary matrix, where $G$ is the total number of gene families and $O$ is the number \ac{otu}s.\par
There are many ways to construct a species tree, but for this project the tree will be constructed with 16S rRNA genes, using Bayesian methods, as implemented in the program MrBayes.
\subsection{Makophylo Rate Estimations}
Given a species tree and a gene family \ac{pa} matrix for the \ac{otu}s of the species tree the R package \textit{markophylo} can provide gene insertion and gene deletion rate estimates \citep{marko}.
The presence or absence of gene families are considered 2 discrete states, for which a $(2\times 2)$ transition rate matrix (of a \ac{mc} model) can be estimated using maximum likelihood techniques.
The values in this estimated transition matrix are the insertion rate (transition probability of gene absence $\to$ presence) and deletion rate (transition probability of gene presence $\to$ absence) \citep{marko}.
\subsection{Network Construction}
Quartet decomposition is method by which \ac{hgt} events can be identified using a set of gene trees and a species tree.
Given a tree $T$ a quartet is a subtree contain 4 of the leaf nodes in $T$, meaning that for a tree with $N$ leaf nodes (or \ac{otu}s) there are $\binom{N}{4}$ unique quartets in that tree.
A quartet $Q$ is considered consistent with a tree if $Q = T|Le(Q)$ where $T|Le(Q)$ is the tree obtained by suppressing all degree-two nodes in $T[X]$ and $T[X]$ is the minimal subtree of T with all nodes in $X$, which is a leaf set of $T$ \citep{hide}.
To calculate the weight of an edge for the network, given a species tree $S$ and a set of gene trees $G$ \citep{hide}:
\begin{enumerate}
    \item Pick a horizontal edge $H = ((u,v),(v,u))$ from $S$
    \item Pick a gene tree $G_i$ in $G$
    \item Decompose $G_i$ into it's set of quartets $\phi_i$
    \item Remove all quartets consistent with $S$ or previously explained from $\phi_i$
    \item Set $RS((u,v),\phi_i)$ to be the number of quartets in $\phi_i$ that support the edge $(u,v)$
    \item Set $NS((u,v),\phi_i)$ to be $RS((u,v),\phi_i)$ divided by $\lambda$, which is the total number of quartets in $S$ that are consistent with the edge $(u,v)$.
    \item The score for the edge $H$ for tree $G_i$ is $max\{NS((u,v),\phi_i),NS((v,u),\phi_i)\}$
    \item The total score for the edge $H$ is the sum of scores for each tree $G_i$
    \item This total score calculation is repeated for each horizontal edge $H_i$ in S, resulting in a list of edges, which is a complete description of the network.
\end{enumerate}
This is further explained in the original work, \citep{hide}.
\subsection{Network Statistics}
All networks will be comprised of nodes representing \ac{otu}s and weighted edges represent the estimated amount of \ac{hgt} events between the two incident nodes.
As multiple sets of networks can be computed for a single set of genera (using different sets of gene trees), bootstrap support for edges and confidence intervals on edge weights can also be calculated.
Given a network, with a set of nodes $V = \{V_0$ \dots $V_i\}$ of cardinality $N$ and a set of weighted edges (an unordered 2-tuple and weight) $T = \{((V_1,V_2),W_{1,2})$ \dots $((V_i,V_j),W_{i,j})\}$ with cardinality $E$ descriptive statistics can be computed as follows \citep{netstat}:
    \begin{itemize}
        \item \textbf{Average Node Degree}: $\frac{1}{|N_u|}\sum_{uv}^{N_u} w_{uv}$ where $N_u$ is the set of nodes incidenent to $u$
        \item \textbf{Average Edge Weight}: $\frac{1}{N_c}\sum_i w_i$, The average edge weight for all nodes with \ac{crsp} or without \ac{crsp}
        \item \textbf{Node Clustering Coefficient}:$\frac{1}{k_u(k_u-1)} \sum_{vw}^{T(u)} (\hat{w}_{uw} \hat{w}_{vw} \hat{w}_{uv})^{\frac{1}{3}}$ where $T(u)$ is the set of traingles containing $u$ \citep{clustering}
        \item \textbf{Node Assortativity}: $A = \frac{Tr(M)-||M^2||}{1-||M^2||}$ Where $M$ is the mixing matrix of a given attribute and $||M||$ is the sum of all elements of $M$. $A \in [-1,1]$.\citep{newmanmix}
        \item \textbf{Network Modularity}: $Q=\frac{1}{2m}\sum_{uv}^W [W_{uv} - \frac{k_u k_v}{2m}]\delta(u,v)$ where $m$ is the total weight of alledges, $k_u$ is the degree of $u$ and $\delta(u,v)$ is 1 if $u$ and $v$ both have or do not have \ac{crsp} systems and 0 otherwise. $Q \in [-1,1]$ \citep{modularity}
        %\item<6-> \textbf{Node Eigenvector Centrality}: $\frac{N-1}{\sum_v d(u,v)}$ where $d(x,y)$ is the length of the shortest path $v \to u$.\citep{egcen}
%        \item<7-> \textbf{Edge Weight KL Divergence}: $D_{KL}(P||Q)= \int_{-\infty}^{\infty} p(x)log(\frac{p(x)}{q(x)})dx$
%        \item<4-> \textbf{Network Diameter}: Shortest path between the 2 furthest nodes.
%        \item<3-> \textbf{Network Density}: $\frac{2E}{N(N-1)}$
    \end{itemize}
Each statistic was computed, either separately for the \ac{crsp} and Non-\ac{crsp} nodes or for the entire network for each of the 1000 bootstraps replicates.
Each replicate was produced with 50 gene trees sampled randomly from all gene trees produced for that genus.
For the statistics estimated separately for \ac{crsp} and Non-\ac{crsp} nodes, the average and standard error was calculated across all replicates.

