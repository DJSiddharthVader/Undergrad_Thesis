%Preamble
%Layout/Spacing
\documentclass[10pt,letter]{article}
\usepackage[margin=1cm]{geometry} %sets margins
\usepackage[compact]{titlesec}%sets margins
\usepackage{setspace} %singles spacing
\usepackage{fullpage} %use entire page
\usepackage{amsmath} %for math command
%Graphics
\usepackage{graphicx} %for inserting pics
\usepackage{pgfplots} %for simple plots
\usepackage[section]{placeins} %for FloatBarrier
\usepackage[linguistics]{forest} %for forests
\usepackage{tikz}% for flow chart
\usetikzlibrary{shapes,arrows,positioning,fit,quotes,shapes.misc}% for flow chart
%Tables
\usepackage{tabularx}
\usepackage{multirow}
\usepackage{array}
%References
\usepackage[natbib=true,style=authoryear,backend=biber,useprefix=true]{biblatex}
\addbibresource{../refs.bib}
\begin{document}
\title{\vspace{-1in}Notes On Papers For Thesis}
\author{Siddharth Reed, 400034828}
\date{\today}
\maketitle
\section*{Estimation Of Gene InDel Rates With Missing Data\cite{indelmiss}}
\subsection*{Intro}
\begin{itemize}
    \item Can infer indel rates from P/A of genes in closely related species
    \item Parsimony underestimates indel rates
    \item Can compare different gene trees as well
    \item Need whole sequence to be sure rearrangements don't mask homologs
    \item ML methods can now account for missing data
    \begin{itemize}
        \item Missing can mean non-whole genomes
        \item Can also be genome reduction beyond normal flux (pathogen deletion)
    \end{itemize}
    \item Evo. rates can vary across lineages, allows for this to some extent
\end{itemize}
\subsection*{Methods}
\begin{itemize}
    \item Use P/A markov chain
    \item Assumes indels are independant, at constant rate
    \item Consider families to avoid paralog issues (cluster above $>$ BLAST \% Ident)
    \item Rate matrix $Q = \begin{bmatrix} -\mu & \mu \\ v & -v \end{bmatrix}$ with insertion,deletion $= v, \mu$ respectively
    \item $P(P^i_d|P^i_a,t) = (\mu + v)^{-1} \cdot (v + \mu e^{-(\mu +v)t})$
    \item using liklihood to accomodate missing data
    \begin{itemize}
        \item $L^i(g_i) = $1 if $g_i$ observed at node $i$, $0$ otherwise
        \item $L^i(A) = (\delta,1)' \implies$ prob of gene missing even if truly present is $(\delta,1)$
        \item $L^i(P) = (1-\delta,o)' \implies$ prob of gene present even if truly absent is $(1-\delta,0)$
        \item $\delta = (\delta_1 \dots \delta_s)$ is the proportion of missing data for all tree members $1 \to s$ where $\delta \in [0,1]$
    \end{itemize}
\end{itemize}
\begin{center}
\scalebox{2.0}{
    \begin{forest}
    for tree={
        if n children=0{
            font=\itshape,
            tier=terminal,
        }{},
    }
    [i
     [j,edge label={node[midway,above,font=\scriptsize]{$t_j$}}]
     [k,edge label={node[midway,above,font=\scriptsize]{$t_k$}}]
    ]
    \end{forest}}
\end{center}
For a given gene and the above tree
$$L_i(g_i) = [\sum_{g_j} p_{g_ig_[j}(t_j)L_j(g_j)] \times [\sum_{g_k} p_{g_ig_k}(t_k)L_k(g_k)]$$
\begin{center}
    \begin{tabular}{c c c c}
                           &                        &\multicolumn{2}{c}{Observed}\\
                           & \multicolumn{1}{c|}{}  & 0          & 1\\\cline{2-4}
     \multirow{2}{*}{True} & \multicolumn{1}{c|}{0} & 1          & 0\\
                           & \multicolumn{1}{c|}{1} & $\delta_i$ & $1 - \delta_i$\\
    \end{tabular}
\end{center}
\begin{itemize}
    \item Pr(observed P/A of $g_i$) = $f($\textbf{$x_h$}$) = \sum_{x_0} \pi_{x_0}L_0(x_0)$
    \item Log-lkl of P/A pattern for a set of genes ($\Theta$) is $l(\Theta) = \sum_{h=1}^{N} \pi_{x_0}L_0(x_0)$
    \item correction for genes never observed (lost over time) $L^h_+ = \frac{l^h}{1-L^h_-}$ ($L^h_-$ is Pr(gene h absent in all taxa, computed bu calculating liklihood of all 0 vector on the tree)
    \item Assumed all genes are equally likely to be missing
    \item Four models used
\end{itemize}
\subsection*{Results}
\begin{itemize}
    \item BIC is able to recover $\mu$ and $v$ params effectively within 100 runs on a set of 5000 genes in 5 taxa in homogenous indel rates
    \item Deletion rates are artificially inflated if missing data not accounted for
    \item simulation procedure:
    \begin{itemize}
        \item generate tree with taxa
        \item branch lengths estimated fro beta dist.(1:4,8), with scaling factor
        \item 500 rnd samples of 5000 phyletic patterns
        \item patterns simulated with $\mu \in [0.625,1.167]$ and $v \in [0.875,2]$,
    \end{itemize}
    \item assumed at least 3 taxa have (no $\delta$)
    \item for \textit{Troy} OTUs, estimated up to $\sim 3$ indel event per base sub. by Model 1
    \item estimating missing data has large effect on param estimates
    \item can cluster branches by length
    \item more stuff
\end{itemize}
\subsection*{Discussion}
\begin{itemize}
    \item Dont overparameterize (no $\delta$ for each tip)
    \item these methods are best for trees of closely related taxa with short branch lengths
    \item extend with GMMs, $\Gamma$ rate var for gene fams
\end{itemize}
\section*{Inferring Horizontal Gene Transfer \cite{ihgt}}
\subsection*{Intro}
\begin{itemize}
    \item Transformation, conjugation, transduction as HGT methods
    \item Methods of detection
    \begin{itemize}
        \item Nuc. composition analysis
        \begin{itemize}
            \item GC content
            \item Codon Usage
        \end{itemize}
        \item only need 1 genome
        \item vanishes over time via mutation
        \item need to account for intragenmic variation
        \item Phylogenetic methods
        \begin{itemize}
            \item Gene vs Species distance (low,high $\implies$ HGT)
            \item Look for ILS between gene/species tree (close genes,farther species)
        \end{itemize}
        \item need a few genomes
        \item genearlly better than parametric methods
        \item can infer donor and time of transfer
        \item generally only applied to coding sequences
        \item issues with duplication $\to$ gene loss vs. HGT
    \end{itemize}
    \item Combining methods can improve results, but also increase FP rate
\end{itemize}
\subsection*{Parametric}
\begin{itemize}
    \item Need HGT candidates to be sig. diff from host signature and insig. diff from donor signature
    \item cant detect ancient transfers as well, mutation
    \item signatures include
    \begin{itemize}
        \item nuc. composition
        \item kmer frequency
        \item codon usage bias
        \item structural features
        \item genomic islands
    \end{itemize}
\end{itemize}
\subsection*{Phylogenetic}
\begin{itemize}
    \item Nodes that look like ILS can indicate HGT
    \item explicit methods
    \begin{itemize}
        \item need strong BS support on these nodes
        \item paralogy can lead to false HGT detection
        \item test if gene/species trees are sig. diff. (KH test), if no resonable explanation infer HGT
        \item spectral approach, compare discordance of gene/species subtrees
        \item create these bipartitions only at strong BS branches, can also use quartet decomposition
        \item If pruning/regrafting can resolve gene/species tree diff., that edit path can imply donor and recipient of HGT
        \item can also use reconciliation methods to map possible HGT events
    \end{itemize}
    \item Genes that are similar in highly diff. species
    \item Implicit methods
    \begin{itemize}
        \item bunch of distance metrics for stuff
        \item also assumptions and issues
    \end{itemize}
\end{itemize}
\section*{Fate of Laterally Transferred Genes \cite{fate}}
\subsection*{Intro}
\begin{itemize}
    \item indels vary between species
    \item inferring gene indels hard
    \item Parsimony underestimates evo. events
    \item assumed insertion = deletion rate for each branch, but branches can vary
    \item suggests many LGT genes are quickly deleted after transfer
    \item LGT rates increase at tree tips
    \item LGT rates generaly $\geq$ than nuc. sub rates
\end{itemize}
\subsection*{Results}
\begin{itemize}
    \item $\alpha,\beta$ were selected using ML, assuming $\alpha = \beta$ and not
    \item $\alpha$ much lower than $\beta$
    \item Based on Bacillus dat, could seee up to 5 gene indels ber nuc. sub
    \item more gene movements in closer vs more distant organisms (i.e. at tips)
    \item Assuming genes cannot be regained has no sig. effect on param estimates
    \item External branches have higher InDel rates than internal $\to$ higher indels at tips
    \item Strain specific genes evolve faster than ancestral genes in other taxa
    \item $K_s$ and $K_ns$ both elevated in strain specific genes (i.e. more recently transfered genes)
\end{itemize}
\subsection*{Discussion}
\begin{itemize}
    \item Want complete genomes to eliminate hidden paralogs or homolog masking
    \item Removed all predicted ORFs with no BLAST homologs
    \item High indel rates not explained by highly mobile genetic elements (patho islands), most such ORFs were excluded
    \item more recent genes have longer branch lengths (evolve faster)
    \item Unlikely just amelieoration as the cause since there is an increase in non-syn mutations
\end{itemize}
\subsection*{Methods}
\begin{itemize}
    \item Phylogeny constructed from 4 catted genes with mrbayes
    \item Model assumes
    \begin{itemize}
        \item insertion, deletion are constant rates
        \item constant number of genes
        \item all indels independatn
        \item insertion = deletion for each branch, can vary across branches
    \end{itemize}
    \item Pr(gene i at node x) =\\
            (Pr(gene stays present)*Pr(gene is present at y) +\\
            Pr(gene stays present)*Pr(gene is present at z)) x\\
            (Pr(gene becomes absent)*Pr(gene is absent at y) +\\
            Pr(gene becomes absent)*Pr(gene is absent at z))\\
            for child nodes y,z
    \item For a given P/A vector, last common ancestor probs are considered equal (Q_i) for each gene
    \item Correction for missing data Q_cor = Q/1-Pr(gene exists but not observed in any taxa)
    \item Root prob Q = gammm(multiply) of Q_cor for each gene
\end{itemize}
\section*{High LGT rates not due to false gene absence\cite{unrate}}
\subsection*{Intro}
\begin{itemize}
    \item Truncated homologs can lead to false inference of gene gain, than multiple gene losses
    \item Since genome sizes are stable over time, gene insertion $\sim=$ deletion rate
    \item failure to recognize truncated genes leads to insertion rate overestimation
    \item considered gene fragment presence (discrete) vs truncation and just full gene presence/absence
    \item full gene model had highest likelihood, meaning truncation not as important as thought
    \item fragments decrease insertion, but increase deletion
    \item ORFans (only in one genome, no NR hits) excluded
    \item Bc genes are much more truncated than non-Bc
    \item more recently trasnfered genes are more likely truncated, consistent with idea that they are transient
    \item truncated genes have higher ka/ks ratios
    \item gene fragments are complicated
\end{itemize}
\section*{Phylogenetic Networks\cite{phynet}}
\subsection*{Phylogenomics}
\begin{itemize}
    \item phylogenies help identify evolutionary events
    \item Less critical genes more likely to be transferred
    \item Networks are commonly used in other fields
\end{itemize}
\subsection*{Networks}
\begin{itemize}
    \item Have nice mathematical properties, useful abstraction for types of analysis
    \item network properties
    \begin{itemize}
        \item weighted
        \item directed
        \item connectivity
    \end{itemize}
    \item Transcription/Metabolic/Gene networks exist
    \item Good for visualizations
\end{itemize}
\subsection*{Phylogenomic Networks}
\begin{itemize}
    \item Start with tree, but allow for horizontal events like LG,recombination, hybridization or genome fusion
    \item Can also apply network theory to regular trees
    \item Split networks
    \begin{itemize}
        \item Built from a bi portion of taxa
        \item Can have compat or incompat splits compared to a tree
    \end{itemize}
    \item Shared microbial transposase network
    \begin{itemize}
        \item indicates LGT contributes more than vertical
        \item Reconstructed with P/A data
        \item integrated habitat data
    \end{itemize}
    \item Can create networks from whole genome data
\end{itemize}
\subsection*{Shared Phylogenomic Network}
\begin{itemize}
    \item Constructed from P/A patters for all pfam orthologs
    \item edges = # of shared genes, vertex = taxa
    \item Construction steps
    \begin{itemize}
        \item pick genomes
        \item group into families
        \item Build PA matrix
        \item sum cols(pfam) for each pair of rows(species) = edge
        \item normalize edge by genome sizes
    \end{itemize}
    \item Frequent chromosome-plasmid links imply LGT
\end{itemize}
\subsection*{LGT Phylogenomic Networks}
\begin{itemize}
    \item Like a shared gene network
    \item vertices = nodes from a tree
    \item edges = putatative LGT events
    \item gene gains are rare, assumed to occur very infrequently
    \begin{itemize}
        \item
        \item
    \end{itemize}
    \item
\end{itemize}
\subsection*{Network Properties and Biological Interpretations}
\begin{itemize}
    \item Connectivity: LGT frequency
    \item Centrality: Nodes that are main LGT drivers
    \item Edge Weight Distribution: Are LGTs bulked or individual?
    \item Diameter: How far genes travel between organisms
    \item Community Structure: Why dense clusters form?
\end{itemize}
\section*{Uncovering Rate Variation of LGT During Bacterial Evolution\cite{unrate}}
\subsection*{Intro}
\begin{itemize}
    \item Overcome parallel indels with max lkl methods
    \item Complexity hypothesis not enough to explain indel rate variation
    \item $\Gamma$ distribution incorporated into indel estimation
    \item Considering rate variation increases likelihood estimations
    \item information content contributes little to rate var param $\alpha$
\end{itemize}
\subsection*{Results}
\begin{itemize}
    \item Same 13 bacillus species used
    \item 3 rate conditions (all =, 2 =, all !=) with/without rate variation
    \item single rates likelihood improved with rate variation param
    \item rate variation also improves 2=,all != model likelihood values
    \item both models support that recently transferred genes have higher indel rates
    \item positive correlation between $\alpha$ and branch length
    \item higher $\alpha$ means lower rate variation
    \item closely related taxa have more rate variation for indels
    \item rate variation is blurry over time
    \item infromational genes have lower indel rates, but they still contribute little to rate cariation
    \item conserved genes contribute more to this rate variation than inormational genes
\end{itemize}
\subsection*{Discussion}
\begin{itemize}
    \item slow evolving genes better for tree construction
    \item more accurate trees from better taxon sampling (more WGs)
    \item tested alternative topologies for trees, got similar $\alpha$ values
    \item gfam thresholds have little effect on gfam size or $\alpha$ values
    \item more rate variation in closer species (lower branch len, lower $\alpha$)
    \item endosymbionts can cause bias, experience faster evolution
    \item rate variation might be a strong local effect because rate variation is higher in recently transferred genes
    \item genes from all categories are subject to transfer
    \item up to $\sim 66%$ of genes may have undergone LGT at one point
    \item incongruencies between close and diverged LGT events may be explained by selective foreign gene retention by bacteria
    \item this selection can be based on condon usage, GC content etc.
    \item Ultimately results are consistent with complexity hypothesis, but much more going on
    \item conserved genes also help explain rate variation
\end{itemize}
\subsection*{Methods}
\begin{itemize}
    \item Used concatenated common gene sequences for tree
    \item select gene trees were generated from literature
    \item duplicated genes removed
    \item super trees construed from incongruent common-select trees
    \item missing data correction
    \item discrete $\Gamma$ model used
    \item estimated indel rates and $\alpha$ that maximized likelihood function given observed gfam pattern data
    \item paralogs excluded
    \item recovered non-annotated genome genes with TBLASTN
    \item ORFans excludede
    \item $< 100$ aa genes excluded
    \item COG with BLASTP was used to identiy informational genes
\end{itemize}
\section*{Does Gene Translocation accelerate LT Gene Evolution\cite{trans}}
\subsection*{Intro}
\begin{itemize}
    \item Recently transferred genes have high evolutionary and turnover rates
    \item little known about association between rearrangements and LGT
    \item translocated genes are more often recently transfered, evolve faster and are more often truncated
    \item or is transfer more likely to result in translocation?
    \item bias or leading strand
\end{itemize}
\subsection*{Methods}
\begin{itemize}
    \item compaare genomes with low rearrangement and similar gene content
    \item identify paralogs via TBLASTN and remove
    \item genes grouped by strain specificity
    \item genes sorted by chromosome location
    \item orthologs with mismatched chr location were deemed translocated
    \item truncated genes considered via BLAST hits with <100\% length match
    \item IS and prophage sequences were identified
    \item Ka/Ks was estimated
\end{itemize}
\subsection*{Results}
\begin{itemize}
    \item Conserved specific genes have higher evo distance and hiher Ka/Ks ratio than conserved nonspecific genes
    \item translocated genes evolve faster than positionally conserved genes
    \item recently transferred genes show most translocation
    \item if translocation is constant, above implies lots of ancestral deletion of transfered genes
    \item excluded genes that are more related to genes in further strains  thatn closer ones
    \item broader spectrum genes (deeper in trees)  are more often translocated
    \item gene truncation is high in translocated genes
\end{itemize}
\subsection*{Discussion}
\begin{itemize}
    \item does accessory genome size grow down tree?
    \item All results are dependant on ortholog identification, but results are robust to different blast cutoffs
    \item high similarity cutoffs $\to$ still see more translocation in recently transferred genes $\implies$ translocation not due to ancestral duplication (would see more divergence from ancestors)
    \item LGT also not likely a confounding factor in positional analysis or ortholog grouping
    \item translocated genes evolve faster (what if they are translocated closer to conserved regions (oriC, termini)?)
    \item often following gene translocation is depletion (truncation, deletion)
    \item translocation can affect function/expression
    \item translocation can be deleterious, may be a mecahnism to increase turnover of recently transferred genes
    \item translocated genes are spatially distributed
    \item IS and prophages may play important part in translocation dynamics
\end{itemize}
\section*{LGT Rates in Prokaryotes: High But Why?\cite{high}}
\subsection*{Quantifying Rate of Change in Gene Content}
\begin{itemize}
    \item Lots of species have high gene indel rates relative to nucleotide indel rates
    \item Changes are short-lived (high tip rates vs low ancestor rates)
    \item Why are most changes transient?
    \begin{itemize}
        \item Most are neutral, lost via drift
        \item Most are deleterious lost via selection
        \item LTG are only beneficial circumstantially, can be lost after serving their short purpose
    \end{itemize}
    \item Most gene indel rate estimated will be underestimated since many deleterious changes in ancestors would have never been observed, especially between distant species
\end{itemize}
\subsection*{Applying mutation rate theory to gene turonver rate}
\begin{itemize}
    \item If LGT is deleterious, why does it happen so much?
    \item LGT depends on (external and internal)
    \begin{itemize}
        \item Natural competency
        \item immune systems (phages,plasmids)
        \item Amount of free-floating DNA around cell
    \end{itemize}
    \item Mutation rate depends on
    \begin{itemize}
        \item Distribution of fitness effects
        \item cost of constraining mutation rate
        \item presence of recombination
    \end{itemize}
    \item MTR is optimized to get most out of mutations without incurring too much fitness/metabolic cost
    \item sexual selection minimizes mutation rate
    \item LGT DFE has higher proportion of strongly beneficial changes than mutation (immediate unction gain)
    \item LGT rates may be high to ensure the uptake of beneficial genes, this would explain high turnover rates as most gene indels would be deleterious and deleted
    \item maybe cost of barrier to LGT is too high to be worth it
\end{itemize}
\section*{Inferring Bacterial Genome Flux Considering Truncated Genes\cite{fluxt}}
\subsection*{Intro}
\begin{itemize}
    \item
    \begin{itemize}
        \item
        \item
    \end{itemize}
    \item
\end{itemize}
\subsection*{Methods}
\begin{itemize}
    \item
    \begin{itemize}
        \item
    \end{itemize}
\end{itemize}
\subsection*{Results}
\begin{itemize}
    \item
    \begin{itemize}
        \item
    \end{itemize}
    \item
\end{itemize}
\subsection*{Discussion}
\begin{itemize}
    \item
\end{itemize}
\section*{Network of Gene Sharing among 329 Proteobacterial Genomes Reveal Differences in LGT Freq at Different Phylogenetic Depth\cite{}}
\subsection*{Intro}
\begin{itemize}
    \item
    \begin{itemize}
        \item
        \item
    \end{itemize}
    \item
\end{itemize}
\subsection*{Methods}
\begin{itemize}
    \item
    \begin{itemize}
        \item
    \end{itemize}
\end{itemize}
\subsection*{Results}
\begin{itemize}
    \item
    \begin{itemize}
        \item
    \end{itemize}
    \item
\end{itemize}
\subsection*{Discussion}
\begin{itemize}
    \item
\end{itemize}
\printbibliography
\end{document}

